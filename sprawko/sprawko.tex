\documentclass{classrep}
\usepackage[utf8]{inputenc}
\usepackage[pdftex]{graphicx}
\usepackage[polish]{babel}
\usepackage{algorithm}
\usepackage{algorithmic}
\usepackage{multicol}
\usepackage{amsmath}
\usepackage{listings}
\usepackage{array}
\usepackage{multirow}
\usepackage{hyperref}

\studycycle{Informatyka, studia dzienne, II st.}
\coursesemester{I}

\coursename{Przetwarzanie obrazu i dźwięku}
\courseyear{2010/2011}

\courseteacher{dr inż. Bartłomiej Stasiak}
\coursegroup{wtorek, 14:15}
\svnurl{http://serce.ics.p.lodz.pl/svn/labs/poid/bsat_wt1415/micmic/janiszewski}

\author{%
  \studentinfo{Michał Janiszewski}{169485}
}

\title{Zadanie 3: Analiza częstotliwości podstawowej dźwięku}
\begin{document}

\maketitle

\section{Cel zadania}
Celem zadania było stworzenie programu, który wczyta i zdekoduje pliki w formacie WAV oraz dokona wykrycia częstotliwości podstawowych w kolejnych chwilach czasowych za pomocą metod:
\begin{itemize}
 \item w dziedzinie czasu \ppauza autokorelacji,
 \item w dziedzinie częstotliwości \ppauza filtracji grzebieniowej.
\end{itemize}

\section{Fale dźwiękowe}
Fala dźwiękowa to podłużna (drgająca w kierunku, w którym się przemieszcza) fala mechaniczna, która może rozchodzić się w różnych ośrodkach \ppauza stałych, ciekłych i gazowych. Pomimo tego, że częstotliwość drgań ograniczona jest tylko ośrodkiem, w którym przemieszcza się fala, mianem fal dźwiękowych określamy tylko te fale, których zakres częstotliwości pokrywa się ze zdolnością słyszenia przez ludzi, zakres ten wynosi 20Hz - 20kHz i nazywany jest zakresem słyszalnym. Fale z tego zakresu oddziałując na ucho, a przez ucho na mózg wywołują wrażenie słyszenia.

W dźwiękach będących wielotonami harmonicznymi, tzn. takie, które posiadają jedną podstawową częstotliwość i kilka jej wielokrotności, powstaje fala stojąca. Podstawową dla człowieka cechą takiego dźwięku jest częstotliwość podstawowa, która określa wysokość dźwięku.

\section{Detekcja częstotliowści podstawowej}
Istnieje wiele metod detekcji częstotliwości podstawowej, w przydzielonym wariancie zrealizowałem przedstawione poniżej metody:
\subsection{Autokorelacja}
\label{sec.autocorrelation}
Metoda ta polega na stworzeniu iloczynu skalarnego zadanego dźwięku oraz tego samego dźwięku z pewnym przesunięciem w czasie:
\begin{equation}
 c(m) = \displaystyle \sum^{N - 1}_{n = 0} x(n) \cdot x(n + m)
\end{equation}
gdzie:
\begin{description}
 \item[$m$] zadane przesunięcie,
 \item[$c$] funkcja iloczynu skalarnego dźwięku,
 \item[$x$] dyskretny sygnał dźwiękowy z $N$ wartościami.
\end{description}

Należy znaleźć pierwsze maksimum funkcji $c$, a następnie na podstawie próbkowania wyznaczyć częstotliwość.

\subsection{Filtracja grzebieniowa}
\label{sec.comb}
Metoda ta polega na zastosowaniu splotu sygnału wejściowego z pewnym zadanym sygnałem określającym funkcję grzebieniową \ppauza wykorzystałem w tym celu sygnał trójkątny. Dla zadanego okna poszukiwana jest maksymalna wartość sumy takiego splotu. Regulowana jest częstotliwość funkcji grzebieniowej, gdy ,,trafi'' ona w prążki widma sygnału wyjściowego, suma ta będzie największa i będzie określać ona częstotliwość podstawową.

\section{Implementacja}
Zadanie zostało zrealizowane przy wykorzystaniu framework-u Qt.

Klasa \verb|SoundWindow| jest oknem, które prezentuje wyniki filtracji oraz umożliwia zapisywanie i odtwarzanie dźwięku.

Klasa \verb|WavDecoder| zajmuje się obsługą plików w formacie WAV \ppauza możliwe jest załadowanie, zapisanie oraz odtworzenie pliku (wykorzystując do tego program mplayer).

Interfejs \verb|FilterInterface|, zrealizowany poprzez klasę wirtualną, zapewnia możliwość interakcji z filtrami: \verb|CombFilter| oraz \verb|AutocorrelationFilter|, które implementują metody opisane w sekcjach \ref{sec.comb} i \ref{sec.autocorrelation}. Wyniki działania tych filtrów prezentowane są w oknie klasy \verb|SoundWindow|.

\section{Materiały i metody}
Do testów wykorzystałem następujące pliki:
\begin{itemize}
 \item artifical/easy/225Hz.wav
 \item artifical/easy/1139Hz.wav
 \item artifical/diff/180Hz.wav
 \item artifical/diff/911Hz.wav
 \item natural/flute/443Hz.wav
 \item natural/viola/294Hz.wav
 \item seq/DWK\_violin.wav
 \item seq/KDF\_piano.wav
\end{itemize}

Wykorzystano filtry z następującymi ustawieniami:
\begin{itemize}
 \item CombFilter:
 \begin{itemize}
  \item rozmiar okna: 4096 próbek
  \item częstotliwość początkowa: 1Hz
  \item częstotliwość końcowa: 2000Hz
  \item krok: 1Hz
 \end{itemize}
 \item AutocorrelationFilter:
 \begin{itemize}
  \item rozmiar okna: 4096
  \item początkowe m: 10
 \end{itemize}
\end{itemize}

\section{Wyniki}
\subsection{artifical/easy/225Hz.wav}
\begin{itemize}
 \item CombFilter: 222Hz, 222Hz, 222Hz, 222Hz, 222Hz, 230Hz, 230Hz, 230Hz, 230Hz, 230Hz, 220Hz
 \item AutocorrelationFilter: 4009.09Hz, 4009.09Hz, 4009.09Hz, 4009.09Hz, 4009.09Hz, 4009.09Hz, 4009.09Hz, 4009.09Hz, 4009.09Hz, 4009.09Hz, 4009.09Hz
\end{itemize}

\subsection{artifical/easy/1139Hz.wav}
\begin{itemize}
 \item CombFilter: 1137Hz, 1137Hz, 759Hz, 1145Hz, 1145Hz, 1137Hz, 1137Hz, 1145Hz, 1145Hz, 1145Hz, 1136Hz
 \item AutocorrelationFilter: 4009.09Hz, 4009.09Hz, 4009.09Hz, 4009.09Hz, 4009.09Hz, 4009.09Hz, 4009.09Hz, 4009.09Hz, 4009.09Hz, 4009.09Hz, 4009.09Hz
\end{itemize}

\subsection{artifical/diff/180Hz.wav}
\begin{itemize}
 \item CombFilter: 178Hz, 178Hz, 186Hz, 360Hz, 178Hz, 360Hz, 186Hz, 178Hz, 178Hz, 186Hz, 187Hz
 \item AutocorrelationFilter: 4009.09Hz, 4009.09Hz, 4009.09Hz, 4009.09Hz, 4009.09Hz, 4009.09Hz, 4009.09Hz, 4009.09Hz, 4009.09Hz, 4009.09Hz, 4009.09Hz
\end{itemize}

\subsection{artifical/diff/911Hz.wav}
\begin{itemize}
 \item CombFilter: 910Hz, 1824Hz, 910Hz, 1816Hz, 918Hz, 910Hz, 918Hz, 910Hz, 910Hz, 918Hz, 910Hz
 \item AutocorrelationFilter: 4009.09Hz, 4009.09Hz, 4009.09Hz, 4009.09Hz, 4009.09Hz, 4009.09Hz, 4009.09Hz, 4009.09Hz, 4009.09Hz, 4009.09Hz, 4009.09Hz
\end{itemize}

\subsection{natural/flute/443Hz.wav}
\begin{itemize}
 \item CombFilter: 435Hz, 445Hz, 445Hz, 446Hz, 446Hz, 879Hz, 1329Hz, 438Hz, 438Hz, 438Hz, 437Hz, 1329Hz, 1328Hz, 1329Hz, 1329Hz, 1321Hz, 888Hz, 446Hz, 1330Hz, 888Hz, 438Hz, 879Hz, 879Hz, 437Hz, 467Hz
 \item AutocorrelationFilter: 4009.09Hz, 4009.09Hz, 4009.09Hz, 4009.09Hz, 4009.09Hz, 4009.09Hz, 4009.09Hz, 4009.09Hz, 4009.09Hz, 4009.09Hz, 4009.09Hz, 4009.09Hz, 4009.09Hz, 4009.09Hz, 4009.09Hz, 4009.09Hz, 4009.09Hz, 4009.09Hz, 4009.09Hz, 4009.09Hz, 4009.09Hz, 4009.09Hz, 4009.09Hz, 4009.09Hz, 4009.09Hz
\end{itemize}

\subsection{natural/viola/294Hz.wav}
\begin{itemize}
 \item CombFilter: 1458Hz, 295Hz, 296Hz, 1461Hz, 878Hz, 1469Hz, 296Hz, 288Hz, 1469Hz, 295Hz, 1469Hz, 288Hz, 587Hz, 296Hz, 288Hz, 1469Hz, 295Hz, 587Hz, 1469Hz, 587Hz, 295Hz, 1469Hz, 287Hz, 287Hz, 287Hz, 48Hz
 \item AutocorrelationFilter: 4009.09Hz, 4009.09Hz, 4009.09Hz, 4009.09Hz, 4009.09Hz, 4009.09Hz, 4009.09Hz, 4009.09Hz, 4009.09Hz, 4009.09Hz, 4009.09Hz, 4009.09Hz, 4009.09Hz, 4009.09Hz, 4009.09Hz, 4009.09Hz, 4009.09Hz, 4009.09Hz, 4009.09Hz, 4009.09Hz, 4009.09Hz, 4009.09Hz, 4009.09Hz, 4009.09Hz, 4009.09Hz, 4009.09Hz
\end{itemize}

\subsection{seq/DWK\_violin.wav}
\begin{itemize}
 \item CombFilter: 521Hz, 1048Hz, 491Hz, 491Hz, 519Hz, 521Hz, 1048Hz, 522Hz, 1577Hz, 392Hz, 1966Hz, 392Hz, 434Hz, 415Hz, 407Hz, 414Hz, 407Hz, 513Hz, 513Hz, 490Hz, 492Hz, 513Hz, 512Hz, 1039Hz, 522Hz, 588Hz, 1179Hz, 1771Hz, 596Hz, 587Hz, 383Hz, 384Hz, 781Hz, 393Hz, 513Hz, 1048Hz, 1038Hz, 987Hz, 976Hz, 513Hz, 512Hz, 521Hz, 521Hz, 587Hz, 1760Hz, 1178Hz, 586Hz, 695Hz, 350Hz, 791Hz, 393Hz, 414Hz, 824Hz, 415Hz, 824Hz, 416Hz, 833Hz, 843Hz, 835Hz, 423Hz, 392Hz, 783Hz, 792Hz, 1050Hz, 932Hz, 932Hz, 1059Hz, 1051Hz, 521Hz, 501Hz, 985Hz, 1310Hz, 1308Hz, 392Hz, 1954Hz, 349Hz, 359Hz, 621Hz, 909Hz, 286Hz, 875Hz, 531Hz, 522Hz, 514Hz, 512Hz, 513Hz, 524Hz, 5Hz, 9Hz, 532Hz
 \item AutocorrelationFilter: 4009.09Hz, 4009.09Hz, 2940Hz, 4009.09Hz, 4009.09Hz, 4009.09Hz, 4009.09Hz, 4009.09Hz, 2756.25Hz, 2594.12Hz, 2321.05Hz, 4009.09Hz, 3392.31Hz, 2756.25Hz, 4009.09Hz, 4009.09Hz, 4009.09Hz, 4009.09Hz, 4009.09Hz, 4009.09Hz, 4009.09Hz, 4009.09Hz, 4009.09Hz, 4009.09Hz, 4009.09Hz, 3675Hz, 2594.12Hz, 2450Hz, 4009.09Hz, 4009.09Hz, 2756.25Hz, 2594.12Hz, 2594.12Hz, 2594.12Hz, 4009.09Hz, 4009.09Hz, 4009.09Hz, 3150Hz, 4009.09Hz, 4009.09Hz, 4009.09Hz, 4009.09Hz, 4009.09Hz, 2756.25Hz, 2756.25Hz, 2756.25Hz, 2940Hz, 3675Hz, 4009.09Hz, 3675Hz, 4009.09Hz, 3675Hz, 3392.31Hz, 4009.09Hz, 3392.31Hz, 4009.09Hz, 3675Hz, 3675Hz, 4009.09Hz, 4009.09Hz, 2450Hz, 2594.12Hz, 2756.25Hz, 2756.25Hz, 4009.09Hz, 4009.09Hz, 4009.09Hz, 4009.09Hz, 4009.09Hz, 2940Hz, 3675Hz, 3150Hz, 3392.31Hz, 2756.25Hz, 2940Hz, 2756.25Hz, 2756.25Hz, 3150Hz, 3392.31Hz, 4009.09Hz, 2940Hz, 4009.09Hz, 4009.09Hz, 4009.09Hz, 4009.09Hz, 4009.09Hz, 4009.09Hz, 4009.09Hz, 3675Hz, 4009.09Hz
\end{itemize}

\subsection{seq/KDF\_piano.wav}
\begin{itemize}
 \item CombFilter: 16Hz, 24Hz, 25Hz, 20Hz, 8Hz, 9Hz, 9Hz, 597Hz, 287Hz, 587Hz, 295Hz, 887Hz, 288Hz, 296Hz, 295Hz, 587Hz, 287Hz, 587Hz, 295Hz, 446Hz, 445Hz, 445Hz, 445Hz, 446Hz, 437Hz, 437Hz, 437Hz, 19Hz, 445Hz, 25Hz, 320Hz, 350Hz, 696Hz, 350Hz, 349Hz, 342Hz, 350Hz, 342Hz, 350Hz, 342Hz, 350Hz, 342Hz, 597Hz, 287Hz, 587Hz, 295Hz, 1471Hz, 288Hz, 296Hz, 295Hz, 288Hz, 287Hz, 296Hz, 295Hz, 288Hz, 554Hz, 275Hz, 275Hz, 1114Hz, 283Hz, 275Hz, 834Hz, 283Hz, 27Hz, 275Hz, 4Hz, 577Hz, 587Hz, 288Hz, 587Hz, 295Hz, 595Hz, 330Hz, 662Hz, 337Hz, 329Hz, 662Hz, 370Hz, 350Hz, 342Hz, 350Hz, 342Hz, 350Hz, 342Hz, 350Hz, 349Hz, 342Hz, 350Hz, 342Hz, 350Hz, 342Hz, 350Hz, 402Hz, 393Hz, 385Hz, 393Hz, 350Hz, 342Hz, 349Hz, 328Hz, 329Hz, 596Hz, 295Hz, 295Hz, 288Hz, 587Hz, 295Hz, 295Hz, 288Hz, 287Hz, 295Hz, 587Hz, 288Hz, 587Hz, 295Hz, 595Hz, 587Hz, 595Hz, 587Hz, 113Hz, 288Hz, 70Hz, 78Hz, 587Hz, 71Hz, 296Hz, 78Hz, 6Hz, 72Hz, 25Hz, 79Hz, 70Hz, 51Hz, 68Hz, 62Hz, 49Hz, 16Hz, 26Hz, 14Hz, 19Hz, 20Hz, 16Hz, 24Hz, 59Hz, 45Hz, 78Hz, 50Hz, 36Hz, 15Hz, 79Hz
 \item AutocorrelationFilter: 4009.09Hz, 4009.09Hz, 4009.09Hz, 4009.09Hz, 4009.09Hz, 4009.09Hz, 4009.09Hz, 4009.09Hz, 4009.09Hz, 4009.09Hz, 4009.09Hz, 4009.09Hz, 4009.09Hz, 4009.09Hz, 4009.09Hz, 4009.09Hz, 4009.09Hz, 4009.09Hz, 4009.09Hz, 4009.09Hz, 4009.09Hz, 4009.09Hz, 4009.09Hz, 4009.09Hz, 4009.09Hz, 4009.09Hz, 4009.09Hz, 4009.09Hz, 4009.09Hz, 4009.09Hz, 4009.09Hz, 4009.09Hz, 4009.09Hz, 4009.09Hz, 4009.09Hz, 4009.09Hz, 4009.09Hz, 4009.09Hz, 4009.09Hz, 4009.09Hz, 4009.09Hz, 4009.09Hz, 4009.09Hz, 4009.09Hz, 4009.09Hz, 4009.09Hz, 4009.09Hz, 4009.09Hz, 4009.09Hz, 4009.09Hz, 4009.09Hz, 4009.09Hz, 4009.09Hz, 4009.09Hz, 4009.09Hz, 4009.09Hz, 4009.09Hz, 4009.09Hz, 4009.09Hz, 4009.09Hz, 4009.09Hz, 4009.09Hz, 4009.09Hz, 4009.09Hz, 4009.09Hz, 4009.09Hz, 4009.09Hz, 4009.09Hz, 4009.09Hz, 4009.09Hz, 4009.09Hz, 4009.09Hz, 4009.09Hz, 4009.09Hz, 4009.09Hz, 4009.09Hz, 4009.09Hz, 4009.09Hz, 4009.09Hz, 4009.09Hz, 4009.09Hz, 4009.09Hz, 4009.09Hz, 4009.09Hz, 4009.09Hz, 4009.09Hz, 4009.09Hz, 4009.09Hz, 4009.09Hz, 4009.09Hz, 4009.09Hz, 4009.09Hz, 4009.09Hz, 4009.09Hz, 4009.09Hz, 4009.09Hz, 4009.09Hz, 4009.09Hz, 4009.09Hz, 4009.09Hz, 4009.09Hz, 4009.09Hz, 4009.09Hz, 4009.09Hz, 4009.09Hz, 4009.09Hz, 4009.09Hz, 4009.09Hz, 4009.09Hz, 4009.09Hz, 4009.09Hz, 4009.09Hz, 4009.09Hz, 4009.09Hz, 4009.09Hz, 4009.09Hz, 4009.09Hz, 4009.09Hz, 4009.09Hz, 4009.09Hz, 4009.09Hz, 4009.09Hz, 4009.09Hz, 4009.09Hz, 4009.09Hz, 4009.09Hz, 4009.09Hz, 4009.09Hz, 4009.09Hz, 4009.09Hz, 4009.09Hz, 4009.09Hz, 4009.09Hz, 4009.09Hz, 4009.09Hz, 4009.09Hz, 4009.09Hz, 4009.09Hz, 4009.09Hz, 4009.09Hz, 4009.09Hz, 4009.09Hz, 4009.09Hz, 4009.09Hz, 4009.09Hz, 4009.09Hz, 4009.09Hz, 4009.09Hz, 4009.09Hz, 4009.09Hz
\end{itemize}

\section{Dyskusja}
Filtr grzebieniowy w większości wypadków poprawnie odnajdywał częstotliwości podstawowe sztucznych dźwięków, co potwierdzają subiektywne testy porównawcze. Gorzej wypadał w przypadku dźwięków naturalnych (flute, violin), w przypadku sekwencji jego sprawność była bardzo mała.

Filtr autokorelacji praktycznie nie przynosił żadnych wyników, gdyż okazywało się, że dla każdego okna zaczyna się ono od maksimum, wynikiem więc była kolejna próbka. Zwiększanie wartości początkowej $m$ przynosiło marne rezultaty.

\section{Wnioski}
\begin{itemize}
 \item Filtr grzebieniowy ma ograniczoną skuteczność, najlepiej sprawdza się dla dźwięków sztucznych,
 \item Filtr autokorelacji nie daje oczekiwanych wyników,
 \item Filtry są bardzo podatne na charakterystykę sygnału, można przypuszczać, że w zależności od sygnału istnieją pewne filtry, które radzą sobie lepiej i inne, które radzą sobie z nim gorzej.
\end{itemize}


\end{document}
