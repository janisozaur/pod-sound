\documentclass{classrep}
\usepackage[utf8]{inputenc}
\usepackage[pdftex]{graphicx}
\usepackage[polish]{babel}
\usepackage{algorithm}
\usepackage{algorithmic}
\usepackage{multicol}
\usepackage{amsmath}
\usepackage{listings}
\usepackage{array}
\usepackage{multirow}
\usepackage{hyperref}

\studycycle{Informatyka, studia dzienne, II st.}
\coursesemester{I}

\coursename{Metody Obliczeniowe Optymalizacji}
\courseyear{2010/2011}

\courseteacher{}
\coursegroup{czwartek, 14:15}
\svnurl{http://serce.ics.p.lodz.pl/svn/labs/poid/bsat_wt1415/micmic}

\author{%
  \studentinfo{Michał Janiszewski}{169485}
}

\title{Zadanie 4: Programowanie liniowe}
\begin{document}

\maketitle

\section{Cel zadania}
Celem zadania było stworzenie programu, który wczyta i zdekoduje pliki w formacie WAV oraz dokona wykrycia częstotliwości podstawowych w kolejnych chwilach czasowych za pomocą metod:
\begin{itemize}
 \item w dziedzinie czasu \ppauza autokorelacji,
 \item w dziedzinie częstotliwości \ppauza filtracji grzebieniowej.
\end{itemize}

\section{Fale dźwiękowe}
Fala dźwiękowa to podłużna (drgająca w kierunku, w którym się przemieszcza) fala mechaniczna, która może rozchodzić się w różnych ośrodkach \ppauza stałych, ciekłych i gazowych. Pomimo tego, że częstotliwość drgań ograniczona jest tylko ośrodkiem, w którym przemieszcza się fala, mianem fal dźwiękowych określamy tylko te fale, których zakres częstotliwości pokrywa się ze zdolnością słyszenia przez ludzi, zakres ten wynosi 20Hz - 20kHz i nazywany jest zakresem słyszalnym. Fale z tego zakresu oddziałując na ucho, a przez ucho na mózg wywołują wrażenie słyszenia.

W dźwiękach będących wielotonami harmonicznymi, tzn. takie, które posiadają jedną podstawową częstotliwość i kilka jej wielokrotności, powstaje fala stojąca. Podstawową dla człowieka cechą takiego dźwięku jest częstotliwość podstawowa, która określa wysokość dźwięku.

\section{Detekcja częstotliowści podstawowej}
Istnieje wiele metod detekcji częstotliwości podstawowej, w przydzielonym wariancie zrealizowałem przedstawione poniżej metody:
\subsection{Autokorelacja}
Metoda ta polega na stworzeniu iloczynu skalarnego zadanego dźwięku oraz tego samego dźwięku z pewnym przesunięciem w czasie:
\begin{equation}
 c(m) = \displaystyle \sum^{N - 1}_{n = 0} x(n) \cdot x(n + m)
\end{equation}
gdzie:
\begin{description}
 \item[$m$] zadane przesunięcie,
 \item[$c$] funkcja iloczynu skalarnego dźwięku,
 \item[$x$] dyskretny sygnał dźwiękowy z $N$ wartościami.
\end{description}

\end{document}
